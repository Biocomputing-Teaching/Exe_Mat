\Exercise Si $\{\overrightarrow{v_1},\overrightarrow{v_2}\} = \{ (2,1),(-2,1) \}$ i $\{\overrightarrow{e_1},\overrightarrow{e_2}\} = \{ (1,0),(0,1) \}$
\begin{enumerate}
  \item Comprova que $\overrightarrow{e_1}$ i $\overrightarrow{e_2}$ són generadors de $\mathbb{R}^2$.
  \item Comprova que $\overrightarrow{e_1}$, $\overrightarrow{v_1}$ i $\overrightarrow{v_2}$ són generadors de $\mathbb{R}^2$.
  \item Ho són $\overrightarrow{v_1}$ i $\overrightarrow{v_2}$?
  \item Dóna exemples de conjunts de vectors d'$\mathbb{R}^2$ que generin altres vectors del mateix espai vectorial amb la forma $\{(\alpha,0):\alpha \in  \mathbb{R} \}$.
  \item Comprova que el conjunt de vectors $\{(1,0,0),(0,1,0),(0,0,1)\}$ genera $\mathbb{R}^3$.
\end{enumerate}

\Answer Els vectors $\overrightarrow{v_1},\overrightarrow{v_2}, \ldots, \overrightarrow{v_n}$ són generadors de l'espai vectorial $E$ al qual pertanyen, i diem $E=\left<\overrightarrow{v_1},\overrightarrow{v_2}, \ldots, \overrightarrow{v_n}\right>$, quan qualsevol $\vec{u}\in E$ es pot posar com a combinació lineal de $\overrightarrow{v_1},\overrightarrow{v_2}, \ldots, \overrightarrow{v_n}$:
\[\vec{u} = \lambda_1 \overrightarrow{v_1} + \lambda_2 \overrightarrow{v_2} + \cdots + \lambda_n \overrightarrow{v_n}\]


\begin{enumerate}
  \item Comprova que $\overrightarrow{e_1}$ i $\overrightarrow{e_2}$ són generadors de $\mathbb{R}^2$.
  \[\vec{u} = \alpha \overrightarrow{e_1} + \beta \overrightarrow{e_2}\]
  o, anàlogament:
  \[
    \systeme*{x=\alpha,y=\beta}
  \]
  Per tant, és obvi que podem trobar valors de $\alpha$ i $\beta$ que satisfacin aquesta expressió per a qualsevol vector $(x,y)\in \mathbb{R}^2$. 
  \item Comprova que $\overrightarrow{e_1}$, $\overrightarrow{v_1}$ i $\overrightarrow{v_2}$ són generadors de $\mathbb{R}^2$.
  \[\vec{u} = \alpha \overrightarrow{e_1} + \beta \overrightarrow{v_1} + \gamma \overrightarrow{v_2}\]
  quedant:
  \[
    \systeme*{x=\alpha+2\beta-2\gamma,y=\beta+\gamma}
  \]
  Donats  valors a $\alpha$, $\beta$ i $\gamma$ podem trobar tots els valors possibles de les components dels vectors $(x,y)$.
  \item Ho són $\overrightarrow{v_1}$ i $\overrightarrow{v_2}$?
  El mateix cas que l'apartat (1).
  \item Dóna exemples de conjunts de vectors d'$\mathbb{R}^2$ que generin altres vectors del mateix espai vectorial amb la forma $\{(\alpha,0):\alpha \in  \mathbb{R} \}$.
  $(1,0)$ o $(-3/2,0)$ serien exemples d'aquests vectors. Cal notar que no generarien tot l'espai $\mathbb{R}^2$, sinó un subespai de dimensió 1 (una recta al pla $\mathbb{R}^2$).
  \item Comprova que el conjunt de vectors $\{(1,0,0),(0,1,0),(0,0,1)\}$ genera $\mathbb{R}^3$.
  Cas anàleg a l'apartat (1).
\end{enumerate}
\blacksquare

