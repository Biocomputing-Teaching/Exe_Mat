\Exercise Trobeu l'equació general de les rectes següents:
\begin{enumerate}
  \item La recta que passa pel punt $(1,2)$ i té per vector director el vector $\uvec{v}=(3,-1)$.
  \item La recta que passa pel punt $(0,0)$ i és paral·lela a la recta $x-y=4$.
  \item La recta que passa pel punt $(1,-2)$ i és perpendicular a la recta $x+2y+5=0$
\end{enumerate}

\Answer Cada cas el podem ractar lleugerament diferent, però totes les opcions són equivalents i bescanviables en la pràctica:
\begin{enumerate}
  \item La recta que passa pel punt $(1,2)$ i té per vector director el vector $\uvec{v}=(3,-1)$.
 
  Comencem amb l'equació contínua de la recta escrivint:
  \[
  \frac{x-1}{3}=\frac{y-2}{-1}
  \]
  D'aquí:
  \begin{eqnarray*}
    -(x-1)&=&3(y-2)\\
    3y+x-7=0
  \end{eqnarray*}

  \item La recta que passa pel punt $(0,0)$ i és paral·lela a la recta $x-y=4$.

  Si les dues rectes són paral·leles, els coeficients de la $x$ i la $y$ en l'equació general seran els mateixos i, per tant, busquem el coeficient independent a l'equació $x-y=C$. Com que la recta passa pel punt $(0,0)$, $C=0$ i la recta demanada és $x-y=0$.

  \item La recta que passa pel punt $(1,-2)$ i és perpendicular a la recta $x+2y+5=0$

  Si la recta problema és ortogonal a $x+2y+5=0$ vol dir que el producte escalar dels seus vectors directors és $0$. El vector director de la recta donada és $(2,1)$ (mira l'equació contínua del primer d'aquests tres apartats) i, per tant, un possible vector ortogonal a aquest seria $(-1,2)$.\footnote{Efectivament, $(2,1)\cdot(-1,2)=0$}. Així, novament com en l'apartat anterior, busquem el coeficient independent a l'equació $2x-2y=C$. Com que la recta problema passa pel punt $(1,-2)$, tenim que $2\cdot1-2\cdot(-2)=6=C$. Per tant, la recta buscada és $2x-2y-6=0$

\end{enumerate}
\blacksquare
