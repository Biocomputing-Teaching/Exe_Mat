
% \usepackage{framed}
% \usepackage{url}
% \usepackage{ifthen}
% \usepackage{longtable}
% \usepackage{fancyvrb}
% \usepackage{cancel}

% \usepackage{lmodern}
\usepackage{amsmath,amssymb,gensymb}%,amsthm,amsfonts,,amscd}
\usepackage[arrowdel]{physics} 
% \usepackage{multirow,booktabs}
\usepackage[dvipsnames,table]{xcolor}
% \usepackage{fullpage}
% \usepackage{lastpage}
\usepackage{pstricks} 
\usepackage{graphicx}
\graphicspath{{../figures/}}
\definecolor{mygreen}{RGB}{28,172,0}
\definecolor{mylilas}{RGB}{170,55,241}
\definecolor{deepblue}{rgb}{0,0,0.5}
\definecolor{deepred}{rgb}{0.6,0,0}
\definecolor{deepgreen}{rgb}{0,0.5,0}

\usepackage{fancyhdr}
% \usepackage{mathrsfs}
% \usepackage{wrapfig}
% \usepackage{setspace}
% \usepackage{calc}
\usepackage{multicol}
\usepackage{listings}



% \usepackage{siunitx}

% \usepackage{cancel}
% %\usepackage[retainorgcmds]{IEEEtrantools}
% \usepackage[margin=3cm]{geometry}
% \newlength{\tabcont}
% \setlength{\parindent}{0.0in}
% \setlength{\parskip}{0.05in}
% \usepackage{empheq}
% \usepackage{framed}
% \usepackage{mdframed}
\usepackage{systeme}

\usepackage{tcolorbox}
% \usepackage{chemfig,chemmacros,chemnum}
% \usepackage{chemformula}

% \usepackage{tcolorbox}
\usepackage{url}
   \let\oldurl\url
\usepackage{hyperref}
   \let\linkurl\url
   \let\url\oldurl


% \RequirePackage{setspace}
% \RequirePackage{lastpage}
% \RequirePackage{extramarks}
% \RequirePackage{chngpage}
% \RequirePackage{soul}


\usepackage{makecell}
\usepackage{caption}
% \usepackage{fancybox}

% %\RequirePackage[text]{amsthm}
% %\RequirePackage{array}
% %\RequirePackage{amscd}
% %\RequirePackage{array}\RequirePackage{dcolumn}
% %\putfig{3.5truein}{PSfig1.3}{Peter's winnings in 40 plays of heads or tails.}{fig 1.3}

% \newcommand{\dx}{\, dx}
% \newcommand{\dy}{\, dy}
% \newcommand{\dt}{\, dt}
% \newcommand{\dth}{\, d\theta}
% \newcommand{\dr}{\, dr}
% \newcommand{\du}{\, du}
\newcommand\uvec[1]{\textbf{#1}}
% \newcommand{\iu}{\hat{\uvec{i}}}
% \newcommand{\ju}{\hat{\uvec{j}}}
% \newcommand{\ku}{\hat{\uvec{k}}}

% \newcommand{\emx}[1]{{\em{#1}\/}}
% \newcommand{\abin}{{\it ab initio}}
% \newcommand{\bs}{\boldsymbol}
% \newcommand{\citenum}{\cite}
% \newcommand{\dGo}{\ensuremath{\Delta G_0}}
% \newcommand{\dG}[2]{\ensuremath{\Delta G_{\rm #1}^{\rm #2}}}
% \newcommand{\dX}[3]{\ensuremath{\Delta #1_{\rm #2}^{\rm #3}}}
% \newcommand{\ddgo}[1]{\ensuremath{\Delta \Delta G_{\rm solv}^{\rm #1}}}
% \newcommand{\ddgstarcat}{\ensuremath{\Delta \Delta g^{\ddagger}_{\rm cat}}}
% \newcommand{\ddgstar}{\ensuremath{\Delta \dgstar}}
% \newcommand{\ddgt}[2]{\ensuremath{\Delta \Delta G_{\rm solv}^{\rm #1, \rm #2}}}
% \newcommand{\ddsstarprime}{\ensuremath{(\Delta \dsstar)'}}
% \newcommand{\deltaepsel}{\ensuremath{\Delta \varepsilon_{\rm el}}}
% \newcommand{\deltaeps}{\ensuremath{\Delta \varepsilon}}
% \newcommand{\dgab}[2]{\ensuremath{\Delta g_{\rm #1}^{\rm #2}}}
% \newcommand{\dga}[1]{\ensuremath{\Delta g_{\rm #1}}}
% \newcommand{\dgb}[1]{\ensuremath{\Delta g^{\rm #1}}}
% \newcommand{\dgcage}{\ensuremath{\Delta g_{\rm cage}}}
% \newcommand{\dgcat}{\ensuremath{\Delta g_{\rm cat}}}
% \newcommand{\dgsoltsatsa}{\ensuremath{\dgsol (\rm TSA)_{\rm TSA}}}
% \newcommand{\dgsoltstsa}{\ensuremath{\dgsol (\rm TS)_{\rm TSA}}}
% \newcommand{\dgsoltsts}{\ensuremath{\dgsol (\rm TS)_{\rm TS}}}
% \newcommand{\dgsol}{\ensuremath{\Delta G_{\rm sol}}}
% \newcommand{\dgstarcage}{\ensuremath{\dgstar_{\rm cage}}}
% \newcommand{\dgstarcat}{\ensuremath{\dgstar_{\rm cat}}}
% \newcommand{\dgstarw}{\ensuremath{\dgstar_{\rm w}}}
% \newcommand{\dgstar}{\ensuremath{\Delta g^{\ddagger}}}
% \newcommand{\dgw}{\ensuremath{\Delta g_{\rm w}}}
% \newcommand{\dg}[2]{\ensuremath{\Delta g_{\rm #1}^{\rm #2}}}
% \newcommand{\dino}{\texttt{DINO}}
% \newcommand{\dsstarcageprime}{\ensuremath{(\dsstarcage)'}}
% \newcommand{\dsstarcage}{\ensuremath{\dsstar_{\rm cage}}}
% \newcommand{\dsstarcatprime}{\ensuremath{(\dsstarcat)'}}
% \newcommand{\dsstarcat}{\ensuremath{\dsstar_{\rm cat}}}
% \newcommand{\dsstarwprime}{\ensuremath{(\dsstarw)'}}
% \newcommand{\dsstarw}{\ensuremath{\dsstar_{\rm w}}}
% \newcommand{\dsstar}{\ensuremath{\Delta S^{\ddagger}}}
% \newcommand{\eg}{{\it e.g.}}
% \newcommand{\etal}{{\it et al.}}
% \newcommand{\gamess}{\texttt{GAMESS}}
% \newcommand{\gauss}{\texttt{GAUSSIAN} 98}
% \newcommand{\golpe}{\texttt{GOLPE}}
% \newcommand{\grid}{\texttt{GRID}}
% \newcommand{\ie}{{\it i.e.}}
% \newcommand{\ith}{{\it i}$^{\rm th}$\ }
% \newcommand{\kbt}{\ensuremath{k_{\rm B} T}}
% \newcommand{\kb}{\ensuremath{k_{\rm B}}}
% \newcommand{\kcage}{\ensuremath{k_{\rm cage}}}
% \newcommand{\kcatkm}{\ensuremath{k_{\rm cat}/K_{\rm M}}}
% \newcommand{\kcat}{\ensuremath{k_{\rm cat}}}
% \newcommand{\km}{kcal mol$^-1$}
% \newcommand{\knon}{\ensuremath{k_{\rm non}}}
% \newcommand{\kw}{\ensuremath{k_{\rm w}}}
% \newcommand{\mepsim}{\texttt{MEPSIM}}
% \newcommand{\mgp}[1]{\marginpar{\scriptsize{#1}}}
% \newcommand{\mipsim}{\texttt{MIPSIM}}
% \newcommand{\mola}{\texttt{MOLARIS}}
% \newcommand{\msms}{\texttt{MSMS}}
% \newcommand{\pdras}{p21$^{\rm ras}$}
% \newcommand{\rgran}{\ensuremath{\mathbb{R}}}
% \newcommand{\rx}[2]{\ensuremath{#1_{\rm #2}}}
% \newcommand{\vs}{{\it vs.}}
% \newcommand{\z}[1]{\ensuremath{\mathbf{#1}}}
% \newcommand{\composed}[2]{#1\mathbin\circ #2}
% \newcommand{\wrt}[1]{{\mbox{\scriptsize w.r.t. \( #1 \)} }}
% \newcommand{\polyspace}{\mathcal{P}}
% \newcommand{\matspace}{\mathcal{M}}
% %\newcommand{\C}{\mathbb{C}}
% \newcommand{\N}{\mathbb{N}}
% \newcommand{\Q}{\mathbb{Q}}
% \newcommand{\Z}{\mathbb{Z}}
% \renewcommand{\Re}{\mathbb{R}}
% \newcommand{\rtres}{\ensuremath{\Re^3}}
% \newcommand{\union}{\cup}
% \newcommand{\dotprod}{\cdot}
% \newcommand*\pkg[1]{\textsf{#1}}

% \newcommand{\trans}[1]{{#1}^{\ensuremath{\mathsf{T}}}} % transpose
% \newcommand{\nbyn}[1]{\ensuremath{#1 \! \times \! #1 }}
% \newcommand{\nbym}[2]{#1 \! \times \! #2 }       % Use in math mode.
% \newcommand{\cat}[2]{#1\!\mathbin{\raise.6ex\hbox{\( {}^\frown \)}}\!#2}
% \newcommand{\generalmatrix}[3]{ %arg1: low-case letter, arg2: rows, arg3: cols
%                \left(
%                   \begin{array}{cccc}
%                     #1_{1,1}  &#1_{1,2}  &\ldots  &#1_{1,#2}  \\
%                     #1_{2,1}  &#1_{2,2}  &\ldots  &#1_{2,#2}  \\
%                               &\vdots                         \\
%                     #1_{#3,1} &#1_{#3,2} &\ldots  &#1_{#3,#2}
%                   \end{array}
%                \right)  }
% \newcommand{\colvec}[1]{\begin{pmatrix} #1 \end{pmatrix}}
% \newcommand{\pr}[1]{\ensuremath{\mathrm{Pr}(#1)}}
% \newcommand{\rep}[2]{ {\rm Rep}_{#2}(#1) }
% \newcommand{\mapsunder}[1]{\stackrel{#1}{\longmapsto}}
% \newcommand{\map}[3]{\mbox{$#1\colon #2\to #3$}}
% \newcommand{\identity}{\mbox{id}}
% \newcommand{\stdbasis}{{\cal E}}
% \newcommand{\sequence}[1]{ \langle#1\rangle }
% \newcommand{\spacer}{\rule[-3mm]{0mm}{8mm}}
% \newcommand{\email}[1]{\url{#1}}
% \newcommand{\zero}{\vec{0}}
% \newcommand{\proj}[2]{\mbox{proj}_{#2}({#1}) }
% %\AtBeginDocument{\newlength{\heightofcdot}
% %\newlength{\widthofcdot}
% %\settoheight{\heightofcdot}{$\cdot$}
% %\settowidth{\widthofcdot}{$\cdot$}
% %\newsavebox{\dotprodcircle}
% %\savebox{\dotprodcircle}{\includegraphics{dotprod.1}}
% %\newcommand{\dotprod}{\mathbin{\raisebox{.5\heightofcdot}{%
% %          \makebox[\widthofcdot]{$\smash{\usebox{\dotprodcircle}}$}}}}}
% \newcommand{\spanof}[1]{\relax [#1\relax ]} % no optional argument!
% \newcommand{\set}[1]{\mbox{$\{#1\}$}} \newcommand{\suchthat}{\bigm|}
% \newcommand{\deter}[1]{ \mathchoice{\left|#1\right|}{|#1|}{|#1|}{|#1|} }
% \newcommand{\secuence}[1]{ \langle#1\rangle }
% \newcommand{\basis}[2]{\secuence{\vec{#1}_1,\ldots,\vec{#1}_{#2}}}



% %--------linsys
% %  Use as \begin{linsys}{3}
% %           x &+ &3y &+ &a &= &7 \\
% %           x &- &3y &+ &a &= &7
% %         \end{linsys}
% % Remark: TeXbook pp. 167-170 says to put a medmuskip around a +; and that's
% % 4/18-ths of an em.  Why does 2/18-ths of an em work?  I don't know, but
% % comparing to a regular displayed equation suggests it is right.
% % (darseneau says LaTeX puts in half an \arraycolsep.)
% \newenvironment{linsys}[2][m]{%
% \setlength{\arraycolsep}{.1111em} % p. 170 TeXbook; a medmuskip
% \begin{array}[#1]{@{}*{#2}{rc}r@{}}
% }{%
% \end{array}}


% header and footer
\pagestyle{fancy}       %                %
\cfoot{\includegraphics[width=2cm]{FCTE}}                %
\rfoot{\thepage}        %
\renewcommand\headrulewidth{0.4pt}   %
\renewcommand\footrulewidth{0.4pt}   %

\usepackage[lastexercise]{exercise}

%numeracions dels llistats
\usepackage{enumitem}
\newlist{llista}{enumerate}{3}
\setlist[llista,1]{label=(\arabic*)}
\setlist[llista,2]{label=(\arabic{llistai}.\arabic*)}
\setlist[llista,3]{label=(\arabic{llistai}.\arabic{llistaii}.\arabic*)}

%%%%%%%%%%%%%%%%%%%%%%%%%%%%%%%%%%%%%%%%%%%%%%
%%%%%%%%%%%%%%%%%%%%%%%%%%%%%%%%%%%%%%%%%%%%%%

% comanda per definir Gauss-Jordan Elimination
\allowdisplaybreaks
\makeatletter
\newcounter{elimination@steps}
\newcolumntype{R}[1]{>{\raggedleft\arraybackslash$}p{#1}<{$}}
\def\elimination@num@rights{}
\def\elimination@num@variables{}
\def\elimination@col@width{}
\newenvironment{elimination}[4][0]
{
    \setcounter{elimination@steps}{0}
    \def\elimination@num@rights{#1}
    \def\elimination@num@variables{#2}
    \def\elimination@col@width{#3}
    \renewcommand{\arraystretch}{#4}
    \start@align\@ne\st@rredtrue\m@ne
}
{
    \endalign
    \ignorespacesafterend
}
\newcommand{\eliminationstep}[2]
{
    \ifnum\value{elimination@steps}>0\sim\quad\fi
    \left[
        \ifnum\elimination@num@rights>0
            \begin{array}
            {@{}*{\elimination@num@variables}{R{\elimination@col@width}}
            |@{}*{\elimination@num@rights}{R{\elimination@col@width}}}
        \else
            \begin{array}
            {@{}*{\elimination@num@variables}{R{\elimination@col@width}}}
        \fi
            #1
        \end{array}
    \right]
    &
    \begin{array}{l}
        #2
    \end{array}
    &%                                    moved second & here
    \addtocounter{elimination@steps}{1}
}
\makeatother

%%%%%%%%%%%%%%%%%%%%%%%%%%%%%%%%%%%%%%%%%%%%%%
%%%%%%%%%%%%%%%%%%%%%%%%%%%%%%%%%%%%%%%%%%%%%%

\usepackage{tikz}
\usepackage{tkz-graph}
\usepackage{pgfplots}
\usepackage{tkz-euclide}
\usetikzlibrary{patterns}
\usetikzlibrary{arrows,automata}
\usetikzlibrary{positioning,calc}%,quotes}
\usetikzlibrary{babel} % solve some problems with different languages like spanish