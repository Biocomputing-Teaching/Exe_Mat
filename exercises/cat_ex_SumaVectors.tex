\Exercise  
Donats el punt $A(1,1)$ i els vectors $\vec{u}=(2,4)$, $\vec{v}=(0,-3)$:
\begin{enumerate}
  \item Aplica al punt $A$ el desplaçament $\vec{u}$, i al nou punt trobat el desplaçament $\vec{v}$. Quin és el vector desplaçament des d'$A$ a la posició final?
  \item Repeteix l'exercici canviant l'ordre dels desplaçaments.
  \item Aplica al punt $A$ el desplaçament $\vec{u}$, i al nou punt trobat el desplaçament $\vec{u}$ novament. Quin és el vector desplaçament des d'$A$ a la posició final?
\end{enumerate}

\Answer

Es tracta de veure que la suma de vectors és commutativa, i que podem multiplicar un vector per un escalar:
\begin{enumerate}
  \item Aplica al punt $A$ el desplaçament $\vec{u}$, i al nou punt trobat el desplaçament $\vec{v}$. Quin és el vector desplaçament des d'$A$ a la posició final?
  \[(A+\vec{u})+\vec{v}=\left((1,1)+(2,4)\right)+(0,-3)=(3,2)\]
  \item Repeteix l'exercici canviant l'ordre dels desplaçaments.
  \[(A+\vec{v})+\vec{u}=\left((1,1)+(0,-3)\right)+(2,4)=(3,2)\]
  \item Aplica al punt $A$ el desplaçament $\vec{u}$, i al nou punt trobat el desplaçament $\vec{u}$ novament. Quin és el vector desplaçament des d'$A$ a la posició final?
  \[(A+\vec{u})+\vec{u}=\left((1,1)+(2,4)\right)+(2,4)=(5,9)=A+2\vec{u}\]
\end{enumerate}
\blacksquare 