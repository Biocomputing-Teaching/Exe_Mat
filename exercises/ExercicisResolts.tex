\documentclass[12pt]{article}

% \usepackage{framed}
% \usepackage{url}
% \usepackage{ifthen}
% \usepackage{longtable}
% \usepackage{fancyvrb}
% \usepackage{cancel}

% \usepackage{lmodern}
\usepackage{amsmath,amssymb,gensymb}%,amsthm,amsfonts,,amscd}
\usepackage[arrowdel]{physics} 
% \usepackage{multirow,booktabs}
\usepackage[dvipsnames,table]{xcolor}
% \usepackage{fullpage}
% \usepackage{lastpage}
\usepackage{pstricks} 
\usepackage{graphicx}
\graphicspath{{../figures/}}
\definecolor{mygreen}{RGB}{28,172,0}
\definecolor{mylilas}{RGB}{170,55,241}
\definecolor{deepblue}{rgb}{0,0,0.5}
\definecolor{deepred}{rgb}{0.6,0,0}
\definecolor{deepgreen}{rgb}{0,0.5,0}

\usepackage{fancyhdr}
% \usepackage{mathrsfs}
% \usepackage{wrapfig}
% \usepackage{setspace}
% \usepackage{calc}
\usepackage{multicol}
\usepackage{listings}



% \usepackage{siunitx}

% \usepackage{cancel}
% %\usepackage[retainorgcmds]{IEEEtrantools}
% \usepackage[margin=3cm]{geometry}
% \newlength{\tabcont}
% \setlength{\parindent}{0.0in}
% \setlength{\parskip}{0.05in}
% \usepackage{empheq}
% \usepackage{framed}
% \usepackage{mdframed}
\usepackage{systeme}

\usepackage{tcolorbox}
% \usepackage{chemfig,chemmacros,chemnum}
% \usepackage{chemformula}

% \usepackage{tcolorbox}
\usepackage{url}
   \let\oldurl\url
\usepackage{hyperref}
   \let\linkurl\url
   \let\url\oldurl


% \RequirePackage{setspace}
% \RequirePackage{lastpage}
% \RequirePackage{extramarks}
% \RequirePackage{chngpage}
% \RequirePackage{soul}


\usepackage{makecell}
\usepackage{caption}
% \usepackage{fancybox}

% %\RequirePackage[text]{amsthm}
% %\RequirePackage{array}
% %\RequirePackage{amscd}
% %\RequirePackage{array}\RequirePackage{dcolumn}
% %\putfig{3.5truein}{PSfig1.3}{Peter's winnings in 40 plays of heads or tails.}{fig 1.3}

% \newcommand{\dx}{\, dx}
% \newcommand{\dy}{\, dy}
% \newcommand{\dt}{\, dt}
% \newcommand{\dth}{\, d\theta}
% \newcommand{\dr}{\, dr}
% \newcommand{\du}{\, du}
\newcommand\uvec[1]{\textbf{#1}}
% \newcommand{\iu}{\hat{\uvec{i}}}
% \newcommand{\ju}{\hat{\uvec{j}}}
% \newcommand{\ku}{\hat{\uvec{k}}}

% \newcommand{\emx}[1]{{\em{#1}\/}}
% \newcommand{\abin}{{\it ab initio}}
% \newcommand{\bs}{\boldsymbol}
% \newcommand{\citenum}{\cite}
% \newcommand{\dGo}{\ensuremath{\Delta G_0}}
% \newcommand{\dG}[2]{\ensuremath{\Delta G_{\rm #1}^{\rm #2}}}
% \newcommand{\dX}[3]{\ensuremath{\Delta #1_{\rm #2}^{\rm #3}}}
% \newcommand{\ddgo}[1]{\ensuremath{\Delta \Delta G_{\rm solv}^{\rm #1}}}
% \newcommand{\ddgstarcat}{\ensuremath{\Delta \Delta g^{\ddagger}_{\rm cat}}}
% \newcommand{\ddgstar}{\ensuremath{\Delta \dgstar}}
% \newcommand{\ddgt}[2]{\ensuremath{\Delta \Delta G_{\rm solv}^{\rm #1, \rm #2}}}
% \newcommand{\ddsstarprime}{\ensuremath{(\Delta \dsstar)'}}
% \newcommand{\deltaepsel}{\ensuremath{\Delta \varepsilon_{\rm el}}}
% \newcommand{\deltaeps}{\ensuremath{\Delta \varepsilon}}
% \newcommand{\dgab}[2]{\ensuremath{\Delta g_{\rm #1}^{\rm #2}}}
% \newcommand{\dga}[1]{\ensuremath{\Delta g_{\rm #1}}}
% \newcommand{\dgb}[1]{\ensuremath{\Delta g^{\rm #1}}}
% \newcommand{\dgcage}{\ensuremath{\Delta g_{\rm cage}}}
% \newcommand{\dgcat}{\ensuremath{\Delta g_{\rm cat}}}
% \newcommand{\dgsoltsatsa}{\ensuremath{\dgsol (\rm TSA)_{\rm TSA}}}
% \newcommand{\dgsoltstsa}{\ensuremath{\dgsol (\rm TS)_{\rm TSA}}}
% \newcommand{\dgsoltsts}{\ensuremath{\dgsol (\rm TS)_{\rm TS}}}
% \newcommand{\dgsol}{\ensuremath{\Delta G_{\rm sol}}}
% \newcommand{\dgstarcage}{\ensuremath{\dgstar_{\rm cage}}}
% \newcommand{\dgstarcat}{\ensuremath{\dgstar_{\rm cat}}}
% \newcommand{\dgstarw}{\ensuremath{\dgstar_{\rm w}}}
% \newcommand{\dgstar}{\ensuremath{\Delta g^{\ddagger}}}
% \newcommand{\dgw}{\ensuremath{\Delta g_{\rm w}}}
% \newcommand{\dg}[2]{\ensuremath{\Delta g_{\rm #1}^{\rm #2}}}
% \newcommand{\dino}{\texttt{DINO}}
% \newcommand{\dsstarcageprime}{\ensuremath{(\dsstarcage)'}}
% \newcommand{\dsstarcage}{\ensuremath{\dsstar_{\rm cage}}}
% \newcommand{\dsstarcatprime}{\ensuremath{(\dsstarcat)'}}
% \newcommand{\dsstarcat}{\ensuremath{\dsstar_{\rm cat}}}
% \newcommand{\dsstarwprime}{\ensuremath{(\dsstarw)'}}
% \newcommand{\dsstarw}{\ensuremath{\dsstar_{\rm w}}}
% \newcommand{\dsstar}{\ensuremath{\Delta S^{\ddagger}}}
% \newcommand{\eg}{{\it e.g.}}
% \newcommand{\etal}{{\it et al.}}
% \newcommand{\gamess}{\texttt{GAMESS}}
% \newcommand{\gauss}{\texttt{GAUSSIAN} 98}
% \newcommand{\golpe}{\texttt{GOLPE}}
% \newcommand{\grid}{\texttt{GRID}}
% \newcommand{\ie}{{\it i.e.}}
% \newcommand{\ith}{{\it i}$^{\rm th}$\ }
% \newcommand{\kbt}{\ensuremath{k_{\rm B} T}}
% \newcommand{\kb}{\ensuremath{k_{\rm B}}}
% \newcommand{\kcage}{\ensuremath{k_{\rm cage}}}
% \newcommand{\kcatkm}{\ensuremath{k_{\rm cat}/K_{\rm M}}}
% \newcommand{\kcat}{\ensuremath{k_{\rm cat}}}
% \newcommand{\km}{kcal mol$^-1$}
% \newcommand{\knon}{\ensuremath{k_{\rm non}}}
% \newcommand{\kw}{\ensuremath{k_{\rm w}}}
% \newcommand{\mepsim}{\texttt{MEPSIM}}
% \newcommand{\mgp}[1]{\marginpar{\scriptsize{#1}}}
% \newcommand{\mipsim}{\texttt{MIPSIM}}
% \newcommand{\mola}{\texttt{MOLARIS}}
% \newcommand{\msms}{\texttt{MSMS}}
% \newcommand{\pdras}{p21$^{\rm ras}$}
% \newcommand{\rgran}{\ensuremath{\mathbb{R}}}
% \newcommand{\rx}[2]{\ensuremath{#1_{\rm #2}}}
% \newcommand{\vs}{{\it vs.}}
% \newcommand{\z}[1]{\ensuremath{\mathbf{#1}}}
% \newcommand{\composed}[2]{#1\mathbin\circ #2}
% \newcommand{\wrt}[1]{{\mbox{\scriptsize w.r.t. \( #1 \)} }}
% \newcommand{\polyspace}{\mathcal{P}}
% \newcommand{\matspace}{\mathcal{M}}
% %\newcommand{\C}{\mathbb{C}}
% \newcommand{\N}{\mathbb{N}}
% \newcommand{\Q}{\mathbb{Q}}
% \newcommand{\Z}{\mathbb{Z}}
% \renewcommand{\Re}{\mathbb{R}}
% \newcommand{\rtres}{\ensuremath{\Re^3}}
% \newcommand{\union}{\cup}
% \newcommand{\dotprod}{\cdot}
% \newcommand*\pkg[1]{\textsf{#1}}

% \newcommand{\trans}[1]{{#1}^{\ensuremath{\mathsf{T}}}} % transpose
% \newcommand{\nbyn}[1]{\ensuremath{#1 \! \times \! #1 }}
% \newcommand{\nbym}[2]{#1 \! \times \! #2 }       % Use in math mode.
% \newcommand{\cat}[2]{#1\!\mathbin{\raise.6ex\hbox{\( {}^\frown \)}}\!#2}
% \newcommand{\generalmatrix}[3]{ %arg1: low-case letter, arg2: rows, arg3: cols
%                \left(
%                   \begin{array}{cccc}
%                     #1_{1,1}  &#1_{1,2}  &\ldots  &#1_{1,#2}  \\
%                     #1_{2,1}  &#1_{2,2}  &\ldots  &#1_{2,#2}  \\
%                               &\vdots                         \\
%                     #1_{#3,1} &#1_{#3,2} &\ldots  &#1_{#3,#2}
%                   \end{array}
%                \right)  }
% \newcommand{\colvec}[1]{\begin{pmatrix} #1 \end{pmatrix}}
% \newcommand{\pr}[1]{\ensuremath{\mathrm{Pr}(#1)}}
% \newcommand{\rep}[2]{ {\rm Rep}_{#2}(#1) }
% \newcommand{\mapsunder}[1]{\stackrel{#1}{\longmapsto}}
% \newcommand{\map}[3]{\mbox{$#1\colon #2\to #3$}}
% \newcommand{\identity}{\mbox{id}}
% \newcommand{\stdbasis}{{\cal E}}
% \newcommand{\sequence}[1]{ \langle#1\rangle }
% \newcommand{\spacer}{\rule[-3mm]{0mm}{8mm}}
% \newcommand{\email}[1]{\url{#1}}
% \newcommand{\zero}{\vec{0}}
% \newcommand{\proj}[2]{\mbox{proj}_{#2}({#1}) }
% %\AtBeginDocument{\newlength{\heightofcdot}
% %\newlength{\widthofcdot}
% %\settoheight{\heightofcdot}{$\cdot$}
% %\settowidth{\widthofcdot}{$\cdot$}
% %\newsavebox{\dotprodcircle}
% %\savebox{\dotprodcircle}{\includegraphics{dotprod.1}}
% %\newcommand{\dotprod}{\mathbin{\raisebox{.5\heightofcdot}{%
% %          \makebox[\widthofcdot]{$\smash{\usebox{\dotprodcircle}}$}}}}}
% \newcommand{\spanof}[1]{\relax [#1\relax ]} % no optional argument!
% \newcommand{\set}[1]{\mbox{$\{#1\}$}} \newcommand{\suchthat}{\bigm|}
% \newcommand{\deter}[1]{ \mathchoice{\left|#1\right|}{|#1|}{|#1|}{|#1|} }
% \newcommand{\secuence}[1]{ \langle#1\rangle }
% \newcommand{\basis}[2]{\secuence{\vec{#1}_1,\ldots,\vec{#1}_{#2}}}



% %--------linsys
% %  Use as \begin{linsys}{3}
% %           x &+ &3y &+ &a &= &7 \\
% %           x &- &3y &+ &a &= &7
% %         \end{linsys}
% % Remark: TeXbook pp. 167-170 says to put a medmuskip around a +; and that's
% % 4/18-ths of an em.  Why does 2/18-ths of an em work?  I don't know, but
% % comparing to a regular displayed equation suggests it is right.
% % (darseneau says LaTeX puts in half an \arraycolsep.)
% \newenvironment{linsys}[2][m]{%
% \setlength{\arraycolsep}{.1111em} % p. 170 TeXbook; a medmuskip
% \begin{array}[#1]{@{}*{#2}{rc}r@{}}
% }{%
% \end{array}}


% header and footer
\pagestyle{fancy}       %                %
\cfoot{\includegraphics[width=2cm]{FCTE}}                %
\rfoot{\thepage}        %
\renewcommand\headrulewidth{0.4pt}   %
\renewcommand\footrulewidth{0.4pt}   %

\usepackage[lastexercise]{exercise}

%numeracions dels llistats
\usepackage{enumitem}
\newlist{llista}{enumerate}{3}
\setlist[llista,1]{label=(\arabic*)}
\setlist[llista,2]{label=(\arabic{llistai}.\arabic*)}
\setlist[llista,3]{label=(\arabic{llistai}.\arabic{llistaii}.\arabic*)}

%%%%%%%%%%%%%%%%%%%%%%%%%%%%%%%%%%%%%%%%%%%%%%
%%%%%%%%%%%%%%%%%%%%%%%%%%%%%%%%%%%%%%%%%%%%%%

% comanda per definir Gauss-Jordan Elimination
\allowdisplaybreaks
\makeatletter
\newcounter{elimination@steps}
\newcolumntype{R}[1]{>{\raggedleft\arraybackslash$}p{#1}<{$}}
\def\elimination@num@rights{}
\def\elimination@num@variables{}
\def\elimination@col@width{}
\newenvironment{elimination}[4][0]
{
    \setcounter{elimination@steps}{0}
    \def\elimination@num@rights{#1}
    \def\elimination@num@variables{#2}
    \def\elimination@col@width{#3}
    \renewcommand{\arraystretch}{#4}
    \start@align\@ne\st@rredtrue\m@ne
}
{
    \endalign
    \ignorespacesafterend
}
\newcommand{\eliminationstep}[2]
{
    \ifnum\value{elimination@steps}>0\sim\quad\fi
    \left[
        \ifnum\elimination@num@rights>0
            \begin{array}
            {@{}*{\elimination@num@variables}{R{\elimination@col@width}}
            |@{}*{\elimination@num@rights}{R{\elimination@col@width}}}
        \else
            \begin{array}
            {@{}*{\elimination@num@variables}{R{\elimination@col@width}}}
        \fi
            #1
        \end{array}
    \right]
    &
    \begin{array}{l}
        #2
    \end{array}
    &%                                    moved second & here
    \addtocounter{elimination@steps}{1}
}
\makeatother

%%%%%%%%%%%%%%%%%%%%%%%%%%%%%%%%%%%%%%%%%%%%%%
%%%%%%%%%%%%%%%%%%%%%%%%%%%%%%%%%%%%%%%%%%%%%%

\usepackage{tikz}
\usepackage{tkz-graph}
\usepackage{pgfplots}
\usepackage{tkz-euclide}
\usetikzlibrary{patterns}
\usetikzlibrary{arrows,automata}
\usetikzlibrary{positioning,calc}%,quotes}
\usetikzlibrary{babel} % solve some problems with different languages like spanish


\begin{document}



\title{Exercicis Resolts \\ \large MATEMÀTIQUES \\ Graus en Biologia i Biotecnologia \\[15pt] \includegraphics[width = 60mm]{FCTE}
\thanks{Adreça electrònica: \texttt{jordi.villa@uvic.cat}}}
\author{Jordi Villà i Freixa}
\date{Darrera modificació: \today}
\maketitle

\tableofcontents

%\listofexercises
\newpage


\begin{ExerciseList}
    \section{Models matemàtics}
    \subsection{Models discrets}
    \Exercise  
Considerem una població de ratolins que es troba en una illa deserta. La població inicial de ratolins és de 100 individus. La taxa de creixement natural anual és del 30\% (\( r = 0.30 \)). 

Es vol determinar el temps necessari perquè la població arribi a 1 milió de ratolins en dos casos:

\begin{itemize}
    \item \textbf{Sense aportacions externes:} Només es considera el creixement natural de la població.
    \item \textbf{Amb aportacions externes:} Cada any arriben 20 ratolins nous a més del creixement natural.
\end{itemize}


\Answer


\textbf{Sense Aportacions Externes}

La població en temps \( t \) es modela amb l'equació de creixement exponencial:
\[
P(t) = P_0 \cdot (1 + r)^t
\]
ja que cada pas implica:
\[
P(t+1) = P(t) \cdot (1 + r)
\]
On:
\begin{itemize}
    \item \( P_0 = 100 \) (població inicial)
    \item \( r = 0.30 \) (taxa de creixement)
    \item \( P(t) = 1{,}000{,}000 \) (població objectiu)
\end{itemize}

Per trobar el temps \( t \) necessari per arribar a 1 milió de ratolins, resolem:
\[
1{,}000{,}000 = 100 \cdot (1.30)^t
\]
\[
10{,}000 = (1.30)^t
\]
Aplicant logaritmes:
\[
\ln(10{,}000) = t \cdot \ln(1.30)
\]
\[
t = \frac{\ln(10{,}000)}{\ln(1.30)} \approx \frac{9.21034}{0.26236} \approx 35.11
\]
Per tant, el temps necessari és aproximadament \textbf{35 anys}.

\textbf{Amb Aportacions Externes}

Amb aportacions externes, la població es modela amb l'equació:
\[
P(t+1) = P(t) \cdot (1 + r) + A
\]
o bé
\[
P(t) = P_0 \cdot (1 + r)^t + A \cdot \frac{(1 + r)^t - 1}{r}
\]
On:
\begin{itemize}
    \item \( P_0 = 100 \) (població inicial)
    \item \( r = 0.30 \) (taxa de creixement)
    \item \( A = 20 \) (aportacions externes)
    \item \( P(t) = 1{,}000{,}000 \) (població objectiu)
\end{itemize}

Utilitzant un enfocament iteratiu, s'ha de trobar el temps \( t \) per al qual la població supera 1 milió de ratolins. En aquest cas, el càlcul és:
\[
P(t) = 100 \cdot (1.30)^t + 20 \cdot \frac{(1.30)^t - 1}{0.30}
\]
Després de calcular iterativament, es troba que el temps necessari és aproximadament \textbf{34 anys}.


El següent gràfic mostra l'evolució de la població amb i sense aportacions externes al llarg del temps:

\begin{minipage}[t]{\linewidth}
  \vspace{-2ex}
  \includegraphics[width=0.8\textwidth]{MalthusRatolins}
  \captionof{figure}{Evolució de la població de ratolins amb i sense aportacions externes. La línia horitzontal indica l'objectiu de 1 milió de ratolins.}
  \label{fig:MalthusRatolins}
\end{minipage}


\blacksquare 
\section{Fonaments i espais vectorials}
\subsection{Els nombres reals}
\input{cat_ex_EqAlgebrTransNoreal.tex}
\subsection{Vectors}
\input{cat_ex_modul1}
\input{cat_ex_modul2}
\input{cat_ex_polar1}
\input{cat_ex_OperacionsVectors1}
  \Exercise Siguin $\uvec{u}=(2,-1)$ i $\uvec{v}=(3,4)$, es demana el següent:
  \begin{llista}
    \item Efectueu les següents operacions:
    \begin{llista}
      \item $\uvec{u}+\uvec{v}$
      \item $2\uvec{u}+3\uvec{v}$
      \item $\uvec{u}-\frac{2}{3}\uvec{v}$
    \end{llista}
    \item Calculeu $\uvec{u}\cdot \uvec{v}$
    \item Determineu $\norm{\uvec{u}}$ i $\norm{\uvec{v}}$
    \item Trobeu l'angle format pels dos vectors
  \end{llista}

  \Answer Seguint l'ordre de les qüestions plantejades:
  \begin{llista}
    \item Fem les operacions:
    \begin{llista}
      \item $\uvec{u}+\uvec{v}=(2,-1)+(3,4)=(5,3)$
      \item $2\uvec{u}+3\uvec{v}=2(2,-1)+3(3,4)=(4,-2)+(9,12)=(13,10)$
      \item $\uvec{u}-\frac{2}{3}\uvec{v}=(2,-1)-\frac{2}{3}(3,4)=(2,-1)-(2,\frac{8}{3})=(0,-\frac{11}{3})$
    \end{llista}
    \item $\uvec{u}\cdot \uvec{v}=(2,-1)\cdot(3,4)=2\cdot3+(-1)\cdot4=6-4=2$
    \item $\norm{\uvec{u}}=\sqrt{2^2+(-1)^2}=\sqrt{5}$; $\norm{\uvec{v}}=\sqrt{3^2+4^2}=\sqrt{25}=5$
    \item L'angle format pels dos vectors es pot obtenir usant:
    \[\cos{\alpha}=\frac{\uvec{u}\cdot \uvec{v}}{\norm{\uvec{u}} \cdot \norm{\uvec{v}}}=\frac{2}{5\sqrt{5}}
    \]
    que correspon a un angle de 1.39 radiants, o bé 79.69\degree.
  \end{llista}
  \blacksquare

\input{cat_ex_NormaVector}
\input{cat_ex_SumaVectors}
\subsection{Espais vectorials}
\input{cat_ex_Neutre}
\input{cat_ex_CombinacioLineal1}
\input{cat_ex_CombinacioLineal2}
\input{cat_ex_DependenciaLineal1}
\input{cat_ex_ConjuntsGeneradors1}
\input{cat_ex_Base2}
\section{Càlcul Matricial}
\subsection{Matrius}
\input{cat_ex_ProdMat} 
\input{cat_ex_EqMat}
\input{cat_ex_Base1}
\input{cat_ex_MatCanviBase}
\subsection{Valors i Vectors propis}
\input{cat_ex_valorvectorpropi2}
\input{cat_ex_valorvectorpropi3}
\input{cat_ex_potenciamatriu1}
\input{cat_ex_valorvectorpropi1}
\section{Geometria}
\subsection{Ortogonalitat}
\input{cat_ex_OrtogonalitatVectors}
\input{cat_ex_UnitariOrtogonal}
\input{cat_ex_Quadrilater}
\subsection{Rectes a $\mathbf{R}^2$}
\input{cat_ex_RectesR21}
\input{cat_ex_EqGenRecta}
\input{cat_ex_RectesTallen}
\input{cat_ex_RectesR22}
\subsection{Rectes i plans a $\mathbf{R}^3$}
\input{cat_ex_EqParamRecta}
\input{cat_ex_PosRelRectes}
\input{cat_ex_CartesianesPlans}
\input{cat_ex_PosRelPlans}
\subsection{Rotacions}
\Exercise Determineu la imatge del punt $(0,-2,0)$ quan el rotem un angle de 90 graus al voltant d'un eix que està sobre el pla $YZ$ i que forma un angle de 60 graus amb l'eix $OY$.

\Answer 

Per trobar la imatge haurem d'aplicar l'expressió 
\begin{equation}
  q_B= qq_Aq^*  
\label{Eq:rotquaternio}
\end{equation}
on $q_A$ és el quaternió que correspjn al punt que volem rotar i $q_B$ és la seva imatge després de la rotació, mentre que $q_R$ és el quaternió de rotació unitari $\hat{q}=q_0+q_1i+q_2j+q_3k$.
Recordem que, donat un quaternió unitari de rotació, l'angle de rotació i l'eix associats venen donats per:
\begin{eqnarray*}
  \theta &=& 2 \arccos{q_0}\\
  \hat{r}&=& \frac{(q_1,q_2,q_3)}{\sqrt{(1-q_0^2)}} = \left(
  \frac{q_1}{\sin{\frac{\theta}{2}}},
  \frac{q_2}{\sin{\frac{\theta}{2}}},
  \frac{q_3}{\sin{\frac{\theta}{2}}}
  \right)
\end{eqnarray*}
L'operació inversa seria trobar un quaternió a partir de l'eix de rotació $\hat{r}$ i l'angle de rotació $\theta$:
\begin{equation}
  q=\cos{\frac{\theta}{2}}+i\sin{\frac{\theta}{2}}\hat{r}_x+j\sin{\frac{\theta}{2}}\hat{r}_y+k\sin{\frac{\theta}{2}}\hat{r}_z
  \label{Eq:quaternio}
\end{equation}

Per tant, primer ens cal determinar el vector unitat al voltant del qual farem la rotació. En aquest exercici es tracta d'un vector que formarà un angle de 60 graus (o $\pi/3$ rad) respecte l'eix $OY$ i que es troba en el pla $YZ$. Per tant, les seves coordenades seran: 
\[
  \hat{r}=\cos{\frac{\pi}{3}} \uvec{j} + \sin{\frac{\pi}{3}}\uvec{k} = \frac{1}{2}  \uvec{j} + \frac{\sqrt{3}}{2} \uvec{k}
\]
Sabent l'angle de rotació $\theta=\pi/2$, podem construir el quaternió de rotació usant l'Equació \ref{Eq:quaternio}:
\begin{eqnarray*}
  q=&=&
  \cos{\frac{\pi}{4}}+j\sin{\frac{\pi}{4}}\frac{1}{2}+k\sin{\frac{\pi}{4}}\frac{\sqrt{3}}{2}\\
  &=& \frac{\sqrt{2}}{2}+\frac{\sqrt{2}}{2}\frac{1}{2}j+\frac{\sqrt{2}}{2}\frac{\sqrt{3}}{2}k=
  \frac{\sqrt{2}}{2}+\frac{\sqrt{2}}{4}j+\frac{\sqrt{6}}{4}k
\end{eqnarray*}
que, com es pot comprovar fàcilment,  és unitari i  té com a quaternió conjugat
\[
  q^* =\frac{\sqrt{2}}{2}-\frac{\sqrt{2}}{4}j-\frac{\sqrt{6}}{4}k
\]

D'altra banda, el quaternió $q_A$ corresponent al punt que hem de rotar es calcula usant directament les seves coordenades per a les parts imaginàries del quaternió i deixant zero la part real:
\[
  q_A=  -2j 
\]

Amb tot això, ja podem trobar el quaternió imatge amb l'equació \ref{Eq:rotquaternio}

\[
  q_B= qq_Aq^*  =  \left(\frac{\sqrt{2}}{2}+\frac{\sqrt{2}}{4}j+\frac{\sqrt{6}}{4}k\right) \cdot 
  ( -2j) \cdot \left(\frac{\sqrt{2}}{2}-\frac{\sqrt{2}}{4}j-\frac{\sqrt{6}}{4}k\right)=\sqrt{3}i-\frac{1}{2}j-\frac{\sqrt{3}}{2}k
\]
Per tant, la imatge del punt donat $(0,-2,0)$ és $\left(\sqrt{3},\frac{1}{2},\frac{\sqrt{3}}{2}\right)$
\blacksquare

\section{Transformacions afins}
\input{cat_ex_TransformacionsAfins}
\input{cat_ex_TransformacionsAfins2}
\input{cat_ex_TransformacionsAfins6}
\input{cat_ex_TransformacionsAfins5}
\input{cat_ex_TransformacionsAfins4}
\input{cat_ex_TransformacionsAfins3}
\input{cat_ex_AfiQuadrilater}
\section{Interpolació}
\input{cat_ex_Interpolacions}
\Exercise
\label{Ex:interpolacio2}

Considereu a $\mathbb{R}^2$ els punts $P_0=(1,2)$, $P_1=(2,5)$, $P_2=(3,7)$ i $P_3=(4,3)$ i els vectors $\overrightarrow{P_0'}=(-1,2)$ i $\overrightarrow{P_1'}=(2,-5)$


Considerant els 4 punts $P_0=(1,1)$, $P_1=(2,5)$, $P_2=(3,4)$ i $P_3=(4,2)$, i recordant que la interpolació de Béziers ens ajuda a construir un spline cúbic seguint $Q_0(t)=T\cdot M \cdot G$ on
$  T= \begin{pmatrix}t^3 & t^2 & t & 1\end{pmatrix}$, $M= \begin{pmatrix}
              -1 & 3 & -3 & 1 \\
              3 & -6 & 3 & 0 \\
              -3 & 3 & 0 & 0 \\
              1 & 0 & 0 & 0
        \end{pmatrix}$ i $G= \begin{pmatrix}
              P_0 \\
              P_1 \\
              P_2 \\
              P_3
        \end{pmatrix}$
\begin{enumerate}
  \item Troba l'expressió paramètrica de l'spline cúbic de Béziers que s'obté entre els extrems $P_0$ i $P_3$ amb punts de control $P_1$ i $P_2$.
  \item Quin és el punt que generem amb aquesta interpolació a mig camí de l'intèrval $t\in[0,1]$?
  \item Grafica tots els punts i una aproximació a la funció interpolada.
\end{enumerate}

\Answer 

\begin{enumerate}
  \item Troba l'expressió paramètrica de l'spline cúbic de Béziers que s'obté entre els extrems $P_0$ i $P_3$ amb punts de control $P_1$ i $P_2$.
Usem l'equació donada $Q_0(t)=T\cdot M \cdot G$ i obtenim:

\begin{eqnarray*}
  Q_0(t)&=&\begin{pmatrix}t^3 & t^2 & t & 1\end{pmatrix}
  \begin{pmatrix}
      -1 & 3 & -3 & 1 \\
      3 & -6 & 3 & 0 \\
      -3 & 3 & 0 & 0 \\
      1 & 0 & 0 & 0
  \end{pmatrix}
  \begin{pmatrix}
      P_0 \\
      P_1 \\
      P_2 \\
      P_3
  \end{pmatrix}\\
  \begin{pmatrix}
    x(t)&
    y(t)
  \end{pmatrix}&=&
  \begin{pmatrix}-t^3+3t^2-3t+1 & 3t^3-6t^2+3t& -3t^3+3t^2&t^3\end{pmatrix}
  \begin{pmatrix}
      1 &1 \\
      2 & 5\\
      3 & 4 \\
      4 & 2
  \end{pmatrix}\\
  \begin{pmatrix}
    x(t)\\
    y(t)
  \end{pmatrix}&=&
  \begin{pmatrix}
    3t+1\\
    4t^3-15t^2+12t+1
  \end{pmatrix}
\end{eqnarray*}

  \item Quin és el punt que generem amb aquesta interpolació a mig camí de l'intèrval $t\in[0,1]$?

A mig camí de l'intèrval $t=[0,1]$ som a $t=\frac{1}{2}$. Per tant:
\[
\begin{pmatrix}
  x(\frac{1}{2})\\
  y(\frac{1}{2})
\end{pmatrix}=
\begin{pmatrix}
  3(\frac{1}{2})+1\\
  4(\frac{1}{2})^3-15(\frac{1}{2})^2+12(\frac{1}{2})+1
\end{pmatrix}=
\begin{pmatrix}
  \frac{5}{2}\\
  \frac{15}{4}
\end{pmatrix}=
\begin{pmatrix}
  2.5\\
  3.75
\end{pmatrix}
\]

  \item Grafica tots els punts i una aproximació a la funció interpolada.

  \begin{center}
  \includegraphics{../figures/interpolaciobeziers2.pdf}
  \end{center}

\end{enumerate}
\Exercise
\label{Ex:interpolacio3}

En la competició FIFA 2023 que organitza Vicjove, l'avatar d'una jugadora del Lluçanès està llançant una falta a 20 metres de la porteria.  Un defensa de 2 metres es troba a mig camí entre el punt de llançament de la falta i la porteria. Si el seu xut té una velocitat inicial que ve donada pel vector $P'_0=(8,7)$, l'escaire és a una alçada de 3 metres i la velocitat d'entrada de la pilota a la porteria ve donada pel vector $P'_1=(5,0)$,  
\begin{enumerate}
  \item podrà superar el defensa?.
  \item Quina és l'alçada màxima que adquireix la pilota?
  \item Grafica tots els punts i una aproximació a la funció interpolada.
\end{enumerate}
NOTA important: La jugadora sospita que el rigor del programa FIFA pel que fa a la física pot ser més que dubtós, però afortunadament va aprovar Matemàtiques de primer de multimèdia i recorda que la interpolació d'Hermite segueix la fòrmula $Q_0(t)=T\cdot M \cdot G$ on
$  T= \begin{pmatrix}t^3 & t^2 & t & 1\end{pmatrix}$, 
$M= \begin{pmatrix}
              2 & -2 & 1 & 1 \\
              -3 & 3 & -2 & -1 \\
              0 & 0 & 1 & 0 \\
              1 & 0 & 0 & 0
        \end{pmatrix}$ i 
        $G= \begin{pmatrix}
              P_0 \\
              P_1 \\
              P'_0 \\
              P'_0
        \end{pmatrix}$

\Answer 

\begin{enumerate}
  \item podrà superar el defensa?.
  

  Comencem per graficar el problema:

\begin{center}
  \includegraphics[width=5cm]{../figures/interpolaciohermite2inicial.pdf}
\end{center}

on es pot observar que $P_0=(0,0)$, $P_1=(20,3)$, $P'_0=(8,7)$ i $P'_1=(5,0)$.

{\em Nota}: és obvi que no es tracta d'un problema clàssic de tir parabòlic, ja que en aquest cas tindríem una velocitat horitzontal invariant. En el present exemple aquesta velocitat es redueix, la qual cosa implica una força contrària al moviment de la pilota amb component horitzontal (fricció del vent) que ha reduït aquesta component del valor inicial de 8 al final (en la posició de la porteria) de 5. De la mateixa manera, com que la component $Y$ de la velocitat és zero en arribar a la porteria, es podria deduir que la pilota just ha arribat al seu valor màxim. Més enllà d'això, obviarem la física del problema i ens centrarem en el problema d'interpolació plantejat (que, de fet, ens donarà una corba parametritzada de tercer grau i no de segon grau com seria esperat del problema físic).

La solució del problema es basa en parametritzar el polinomi que descriu el moviment. Segons la informació que ens donen sobre com descriure una interpolació cúbica d'Hermite, tenim:

\begin{eqnarray*}
  Q_0(t)&=&\begin{pmatrix}t^3 & t^2 & t & 1\end{pmatrix}
  \begin{pmatrix}
    2 & -2 & 1 & 1 \\
    -3 & 3 & -2 & -1 \\
    0 & 0 & 1 & 0 \\
    1 & 0 & 0 & 0
\end{pmatrix}
  \begin{pmatrix}
      P_0 \\
      P_1 \\
      P'_0 \\
      P'_1
  \end{pmatrix}\\
  \begin{pmatrix}
    x(t)&
    y(t)
  \end{pmatrix}&=&
  \begin{pmatrix}2t^3-3t^2+1 & -2t^3+3t^2&  t^3-2*t^2+t&t^3-t^2\end{pmatrix}
  \begin{pmatrix}
      0 &0 \\
      20 & 3\\
      8 & 7 \\
      5 & 0
  \end{pmatrix}\\
  \begin{pmatrix}
    x(t)&
    y(t)
  \end{pmatrix}&=&
  \begin{pmatrix}
    -27 t^3 +39 t^2 +8t&
    t^3-5t^2+7t
  \end{pmatrix}
\end{eqnarray*}

Aquesta interpolació es pot expressar com 

\[
\sigma(t)= (-27 t^3 +39 t^2 +8t) \uvec{i} +  (t^3-5t^2+7t)  \uvec{j}
\]

amb derivada

\[
\sigma'(t) = (-81t^2 + 78 t +8) \uvec{i} + (3t^2-10t+7) \uvec(j)  
\]

i és fàcil veure que es correspon amb les dades donades:

\begin{eqnarray*}
  P_0&=&\sigma(0) = 0 \uvec{i} + 0 \uvec{j}\\
  P_1&=&\sigma{1} = (-27+39+8) \uvec{i} + (1-5+7) \uvec{j} = 20 \uvec{i} + 3 \uvec{j}\\
  P'_0&=&\sigma(0) = 8 \uvec{i} + 7 \uvec{j}\\
  P'_1&=&\sigma'{1} = (-81+78+8) \uvec{i} + (3-10+7) \uvec{j} = 5 \uvec{i} + 0 \uvec{j}
\end{eqnarray*}

Per saber si supera el defensa, hem de veure quin valor de $y$ té la funció quan la $x=10$.

Això es pot aconseguir trobant el valor de $t$ per al qual $x=10$:
\[-27 t^3 +39 t^2 +8t=10\]
Solucionar una equació de tercer grau queda fora del coneixement d'aquest exercici, però podem mirar de trobar valors de $t$ que ens acotin el valor de $x$ i, d'aquesta manera, de $y$.

Per exemple, en primera aproximació, si $t=1/2$ tenim
\[-27 \left(\frac{1}{2}\right)^3 +39  \left(\frac{1}{2}\right)^2 +8\left(\frac{1}{2}\right)=-\frac{27}{8}+\frac{39}{4}+\frac{8}{2}=\frac{-27+78+32}{8}=\frac{83}{8}\approx 10\]

Per tant, $t=1/2$ sembla una bona aproximació inicial al temps que tarda la pilota en assolir $x=10$. En aquest valor, tenim:
\[
  y(t=1/2)=\left(\frac{1}{2}\right)^3-5\left(\frac{1}{2}\right)^2+7\frac{1}{2}=\frac{1}{8}-\frac{5}{4}+\frac{7}{2}=\frac{1-10+28}{8}=\frac{19}{8}>2
\]
NOTA: un simple càlcul amb matlab ens mostra que el $(x,y)=(10,2.34)$ quan $t=0.486$, o sigui que la primera aproximació feta és prou bona.

  \item Quina és l'alçada màxima que adquireix la pilota?

  Per trobar l'alçada màxima només cal mirar quan la primera derivada de la component y de la trajectòria es fa zero

  \[
    3t^2-10t+7=0\]
    que succeeix quan: 
    \[
      t=\frac{10 \pm \sqrt{100-4\cdot 3\cdot 7}}{6}\]
    és a dir, a $t_1=7/3$ i a $t=1$. El primer resultat queda exclços perqu`+e és fora del rang de valors de $t$. El segon és el correcte i lliga amb la intuició inicial de que la pilota arribarà al seu màxim just quan toca el travesser. 

  \item Grafica tots els punts i una aproximació a la funció interpolada.
  
  el polinomi resultant serà

\begin{center}
  \includegraphics[width=5cm]{../figures/interpolaciohermite2final.pdf}
\end{center}
\end{enumerate}

\section{Proporció i Tales}
\Exercise

La projecció d'un catet sobre la hipotenusa d'un triangle rectangle medeix 4cm, i la hipotenusa 9c. Quant medeix el catet?

\Answer

Els triangles $AXC$ i $CXB$ de la figura són semblants:
\begin{center}
\begin{tikzpicture}
    \coordinate [label={180:$A$}] (a) at (0,0);
    \coordinate [label={0:$B$}] (b) at (9,0);
    \coordinate [label={-90:$X$}] (x) at (4,0);

    \draw [dashed] let \p1 = ($(b)-(a)$) in [name path=pa] (a) arc (180:0:\x1/2);
    \draw [dashed] let \p1 = ($(b)-(a)$) in [name path=px] (x) |- ($(x)+(90:\x1/2)$);
    \draw [name intersections={of=pa and px}] (a) -- (intersection-1) node [label={90:$C$}] {} -- (b) -- cycle;
    \path [|-|] ($(a)+(0,-1)$) edge node[below] {$4cm$}  ($(x)+(0,-1)$);
    \path [|-|] ($(a)+(0,-2)$) edge node[below] {$9cm$}  ($(b)+(0,-2)$);
  \end{tikzpicture}
\end{center}

  Per tant, es compleix que:
  \begin{eqnarray*}
    \frac{AX}{CX}&=&\frac{CX}{XB}\\
    \frac{4cm}{CX}&=&\frac{CX}{5cm}
  \end{eqnarray*}
  Amb la qual cosa, l'alçada del triangle és $CX=\sqrt{20}=2\sqrt{5}cm$. D'aquí, usant el teorema de Pitàgoras: $AC=\sqrt{4^2+(2\sqrt{5})^2}=\sqrt{16+20}=6cm$.

  També podem notar que, de fet, els tres triangles $AXC$, $CXB$ i $ACB$ de la figura són semblants (només cal observar els seus angles). Per tant:
  \[
    \frac{AC}{AB}=\frac{AX}{AC}  
  \]
  És a dir que: $AC=\sqrt{AB\cdot AX}=6cm$.
\Exercise

Si agafo un paper rectangular que fa 8cm d'ample per 12cm de llarg i li faig un plec que dugui una punta del rectangle a la meitat del costat oposat llarg, quina és la mida del plegament?
%https://twitter.com/zhigangsuo/status/1627722962122993664

\Answer

Dibuixem el problema i marquem els punts importants que defineixin els triangles a comparar:
\begin{center}
\begin{tikzpicture}
    \coordinate [label={225:$A$}] (a) at (0,0);
    \coordinate [label={135:$B$}] (b) at (0,8);
    \coordinate [label={90:$C$}] (c) at (6,8);
    \coordinate [label={-90:$D$}] (d) at (8.34,0);
    \coordinate [label={180:$E$}] (e) at (0,6.25);
    \coordinate [label={45:$F$}] (f) at (12,8);
    \coordinate [label={-45:$G$}] (g) at (12,0);
    \coordinate [label={180:$O$}] (o) at (3,4);

    \path [dashed] (a) edge (e);
    \path (e) edge (b); 
    \path [dashed] (a) edge (d);
    \path [dashed] (a) edge (c);
    \path  (e) edge node[left] {$x$} (d);
    \path  (e) edge (c);
    \path  (b) edge node[above] {$6cm$}  (c);
    \path  (c) edge node[above] {$6cm$}  (f);
    \path  (c) edge (d);
    \path  (f) edge node[right] {$8cm$}  (g);
    \path  (d) edge (g);

  \end{tikzpicture}
\end{center}

Observem que $BC=6cm$, $AC=\sqrt{AB^2+BC^2}=\sqrt{8^2+6^2}=10cm$. Per tant, $AO=OC=5cm$.

Observem també que els triangles $ABC$, $AOD$ i $EOA$ són semblants (per comprovar-ho, agafa un paper rectangulr qualsevol i juga amb possibles plecs similars al proposat). Usant aquestes relacions obtenim que:
\[
  \frac{5}{OD}=\frac{6}{8}  
\]
d'on, $OD=\frac{20}{3}$. I, d'altra banda:
\[
  \frac{5}{OE}=\frac{8}{6}  
\]
d'on, $OE=\frac{15}{4}$.

Per tant, la distància demanada és
\[
  x=ED=OE+OD=\frac{20}{3}+\frac{15}{4}=\frac{125}{12}  
\]
\section{Simetria}
\Exercise Trobeu la simetria d'una lletra "A" majúscula Arial\cite{pere_cruells_matematiques_nodate}:

%https://openaccess.uoc.edu/bitstream/10609/78485/12/Matem%C3%A0tiques%20per%20a%20multim%C3%A8dia%20I_M%C3%B2dul%207_Simetria%20i%20disseny.pdf

\begin{center}
\includegraphics[scale=0.4]{A}
\end{center}

\Answer Primer identifiquem les isometries (elements de simetria que deixen la figura invariant):

\begin{itemize}
  \item La identitat $Id$.
  \item La simetria respecte l'eix verital $\sigma_v$.
  \item Cap rotació en aquest cas.
\end{itemize}

Construïm ara la taula amb les possibles composicions:

\begin{center}
\begin{tabular}{|>{\columncolor{gray}}c|c|c|}
  \hline
  \rowcolor{gray}
  $\circ$         & $Id$          & $\sigma_v$      \\\hline
  $Id$            & $Id$          & $\sigma_v$      \\\hline
  $\sigma_v$      & $\sigma_v$    & $Id$            \\\hline
\end{tabular}
\end{center}

Es tracta, doncs, d'un objecte de simetria $m$ en la notació de Hermann-Mauguin o $D_2$ en la de Schönflies.


\input{cat_ex_SimetriaMercedes}
\end{ExerciseList}
\newpage
\section{Material pràctic}

\begin{itemize}
    \item La Figura \ref{Fig:unitcircle} conté informació sobre els sinus i cosinus d'alguns dels valors d'angles més comuns en els exercicis de l'assignatura.
\end{itemize}

\begin{figure}
    \begin{minipage}[r]{0.7\textwidth}
      \includegraphics[width=\textwidth]{unitcircletikz}
    \end{minipage}\hfill
    \begin{minipage}[l]{0.3\textwidth}
      \caption{
        Esquema dels valors de $\sin x$ i $\cos x$ per a alguns valors d'angles.
      } \label{Fig:unitcircle}
    \end{minipage}
  \end{figure}

\bibliographystyle{plain}
\bibliography{../common/refs}

\end{document}
